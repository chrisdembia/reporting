\documentclass[
%pdf,
%noFooter,
%draft,
%nocolorBG,
%slideBW,
%colorBG,
%slideColor,
%ntnu
%winter
%autumn
]{beamer}

\usepackage{times}
%\usetheme{PaloAlto}
\usetheme{Goettingen}
%\usetheme{Bergen}
%\usetheme{Hannover}
%\usetheme{Rochester}
%\usecolortheme{seahorse} % Simple and clean template
%\usecolortheme{beetle}
%\usecolortheme{beaver}
\usecolortheme{dove}
\setbeamertemplate{footline}[page number]
%\setbeamercovered{transparent}
%\setbeamercovered{invisible}
% To remove the navigation symbols from 
% the bottom of slides%
\setbeamertemplate{navigation symbols}{} 
%
\usepackage{graphicx}
\usepackage{comment}
%\usepackage{bm}         % For typesetting bold math (not \mathbold)
%\logo{\includegraphics[height=0.6cm]{yourlogo.eps}}
%
\title[Hip Spring]
    {Analysis of a simple hip spring using Hamner's running simulations}
\author{Christopher Lee Dembia}
\institute[Stanford]
{
    Neuromuscular Biomechanics Lab \\
    Stanford University \\
    %\medskip
    %{\emph{dembia@stanford.edu}}
}
\date{\today}

\begin{document}

\begin{frame}
    \begin{comment}
    TODO big representative picture.
    \end{comment}
\titlepage
\end{frame}

\section{Motivation}
\begin{frame}
\frametitle{Goal}
\begin{block}
{Why Beamer?}
Does anybody need an introduction to Beamer? I don't think so.
\end{block}
\end{frame}

\section{State of the art}
\begin{frame}
    \frametitle{State of the art}
\end{frame}

\section{Methods}
\begin{frame}
    \frametitle{Methods}
\end{frame}

\section{Expected results and impact}

\begin{frame}
    \frametitle{Hip joint torque: natural versus spring}
\end{frame}

% TODO present final results first and details later, or vice versa?
\begin{frame}
    \begin{comment}
    Sum of activations squared versus percent gait cycle

    show a sample of 4 or 6 gait cycles.
    \end{comment}

    \frametitle{Sum of activations squared (SAS) benefits for early stance, but
    is hit in late stance TODO}
\end{frame}

\begin{frame}
    \begin{comment}
    Integrated sum of activations squared versus spring stiffness
    \end{comment}

    \frametitle{Integrated SAS is minimized for a spring stiffness of TODO}
\end{frame}

\begin{frame}
    \begin{comment}
    Cost of transport versus spring stiffness
    \end{comment}

    \frametitle{Cost of transport is minimized for a spring stiffness of TODO}

\end{frame}

\begin{frame}
    \begin{comment}
    A comparison of the SAS and COT metrics.
    \end{comment}

    \frametitle{Integrated SAS and COT agree on the optimal spring stiffness}
\end{frame}

\begin{frame}
    \begin{comment}
    The consistency of results across the subjects/simulations.
    \end{comment}

    \frametitle{The (1) COT savings and (2) optimal spring stiffness are
    consistent across subjects/simulations}
\end{frame}
 
% --- Questions.
\begin{frame}
    \begin{comment}
    TODO place in quintessential results image/figure.
    \end{comment}
    \centerline{Questions?}
\end{frame}

\begin{frame}
    \frametitle{References}
    \footnotesize{
    \begin{thebibliography}{99}
        \bibitem[Label1, 2010]{key1} Author's name (1987)
            \newblock Title of the paper.
            \newblock \emph{Journal Name} 55(4), 765 -- 799.
    \end{thebibliography}
    }
\end{frame}

\end{document}

%\begin{frame}
%\titlepage
%\end{frame}
%
%\begin{frame}
%\frametitle{Motivation}
%\begin{block}
%{Why Beamer?}
%Does anybody need an introduction to Beamer? I don't think so.
%\end{block}
%\end{frame}
%
%\begin{frame}
%\frametitle{Example of a Theorem}
%\begin{theorem}
%The quick brown fox jumps over the lazy dog.
%\end{theorem}
%\end{frame}
%
%\begin{frame}[fragile] % Notice the [fragile] option %
%\frametitle{Verbatim}
%\begin{example}[Putting Verbatim]
%\begin{verbatim}
%\begin{frame}
%\frametitle{Outline}
%\begin{block}
%{Why Beamer?}
%Does anybody need an introduction to Beamer?
%I don't think so.
%\end{block}
%% Extra carriage return causes problem with verbatim %
%\end{frame}\end{verbatim} 
%\end{example}
%\end{frame}
% 
%\begin{frame}[fragile]  % notice the fragile option, since the body
%    % contains a verbatim command
%    Example of the \verb|\cite| command to give a reference is below:
%    Example of citation using \cite{key1} follows on.
%\end{frame}
% 
%\begin{frame}
%    \frametitle{References}
%    \footnotesize{
%    \begin{thebibliography}{99}
%        \bibitem[Label1, 2010]{key1} Author's name (1987)
%            \newblock Title of the paper.
%            \newblock \emph{Journal Name} 55(4), 765 -- 799.
%    \end{thebibliography}
%    }
%\end{frame}
% 
%\begin{frame}
%    \centerline{The End}
%\end{frame}
%% End of slides
%
